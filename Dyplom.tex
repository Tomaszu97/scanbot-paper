\documentclass[a4paper,onecolumn,oneside,12pt,extrafontsizes]{memoir}
\usepackage[utf8]{inputenc}
\usepackage[T1]{fontenc}
\usepackage[polish]{babel}
\usepackage{setspace}
\usepackage{tabularx}
\usepackage{color,calc}
\usepackage{ebgaramond}
\usepackage{tgtermes}   
\renewcommand*\ttdefault{txtt}
\usepackage{listings}
%%My Edits%%%%%%%%%%%%%%%%%
\usepackage{amsmath}
\usepackage{float}
\usepackage{amssymb}
\usepackage{minted}

\usepackage[ruled,vlined]{algorithm2e}

%%%%%%%%%%%%%%%%%%%%%%%%%%%
\lstset{literate=%-
{ą}{{\k{a}}}1 {ć}{{\'c}}1 {ę}{{\k{e}}}1 {ł}{{\l{}}}1 {ń}{{\'n}}1 {ó}{{\'o}}1 {ś}{{\'s}}1 {ż}{{\.z}}1 {ź}{{\'z}}1 {Ą}{{\k{A}}}1 {Ć}{{\'C}}1 {Ę}{{\k{E}}}1 {Ł}{{\L{}}}1 {Ń}{{\'N}}1 {Ó}{{\'O}}1 {Ś}{{\'S}}1 {Ż}{{\.Z}}1 {Ź}{{\'Z}}1 
    {Ö}{{\"O}}1
    {Ä}{{\"A}}1
    {Ü}{{\"U}}1
    {ß}{{\ss}}1
    {ü}{{\"u}}1
    {ä}{{\"a}}1
    {ö}{{\"o}}1
    {~}{{\textasciitilde}}1
		{—}{{{\textemdash} }}1
}
\newcommand{\listingcaption}[1]
{
\vspace*{\abovecaptionskip}\small 
\refstepcounter{lstlisting}\hfill
Listing \thelstlisting: #1\hfill
\addcontentsline{lol}{lstlisting}{\protect\numberline{\thelstlisting}#1}
}
\lstset{
  breaklines=true,
  postbreak=\mbox{\textcolor{red}{$\hookrightarrow$}\space},
	belowskip=.5\baselineskip
}
\renewcommand\lstlistlistingname{Spis listingów}
\makeatletter
\g@addto@macro\insertchapterspace{\addtocontents{lol}{\protect\addvspace{10pt}}}
\renewcommand*{\l@lstlisting}{\@dottedtocline{1}{0em}{2.3em}}
\makeatother
\renewcommand*{\lstlistlistingname}{Spis listingów} \newlistof{lstlistoflistings}{lol}{\lstlistlistingname}
\clubpenalty=10000
\widowpenalty=10000
\righthyphenmin=3
\renewcommand{\topfraction}{0.95}
\renewcommand{\bottomfraction}{0.95}
\renewcommand{\textfraction}{0.05}
\renewcommand{\floatpagefraction}{0.35}
\setlength{\headsep}{10pt} 
\setlength{\headheight}{13.6pt}
\setlength{\footskip}{\headsep+\headheight}
\setlength{\uppermargin}{\headheight+\headsep+1cm}
\setlength{\textheight}{\paperheight-\uppermargin-\footskip-1.5cm}
\setlength{\textwidth}{\paperwidth-5cm}
\setlength{\spinemargin}{2.5cm}
\setlength{\foremargin}{2.5cm}
\setlength{\marginparsep}{2mm}
\setlength{\marginparwidth}{2.3mm}
\checkandfixthelayout[fixed]
\linespread{1}
\setlength{\parindent}{14.5pt}
\usepackage{memlays}
\usepackage{printlen}
\uselengthunit{pt}
\makeatletter
\newcommand{\showFontSize}{\f@size pt}
\makeatother
\usepackage{enumitem}
\setlist{noitemsep,topsep=4pt,parsep=0pt,partopsep=4pt,leftmargin=*}
\setenumerate{labelindent=0pt,itemindent=0pt,leftmargin=!,label=\arabic*.}
\setlistdepth{4}
\setlist[itemize,1]{label=$\bullet$}
\setlist[itemize,2]{label=\normalfont\bfseries\textendash}
\setlist[itemize,3]{label=$\ast$}
\setlist[itemize,4]{label=$\cdot$}
\renewlist{itemize}{itemize}{4}
\makeatletter
\renewenvironment{quote}{
	\begin{list}{}
	{
	\setlength{\leftmargin}{1em}
	\setlength{\topsep}{0pt}%
	\setlength{\partopsep}{0pt}%
	\setlength{\parskip}{0pt}%
	\setlength{\parsep}{0pt}%
	\setlength{\itemsep}{0pt}
	}
	}{
	\end{list}}
\makeatother
\DisemulatePackage{imakeidx}
\usepackage[makeindex,noautomatic]{imakeidx}
\makeindex
\makeatletter
\makeatother
\usepackage{ifpdf}
\ifpdf
 \usepackage[pdftex,bookmarks,breaklinks,unicode]{hyperref}
 \usepackage[pdftex]{graphicx}
 \DeclareGraphicsExtensions{.pdf,.jpg,.mps,.png}
\pdfcompresslevel=9
\pdfoutput=1
\makeatletter
\AtBeginDocument{
  \hypersetup{
	pdfinfo={
    Title = {\@title},
    Author = {\@author},
    Subject={},
    Keywords={słowa kluczowe},  
		Producer={producer},
		Creator={pdftex}
	}}
}
\pdftrailerid{}
\pdfsuppressptexinfo15

\makeatother
\else
\usepackage{graphicx}
\DeclareGraphicsExtensions{.eps,.ps,.jpg,.mps,.png}
\fi
\sloppy
\setcounter{secnumdepth}{2}
\setcounter{tocdepth}{2}
\setsecnumdepth{subsection}
\makeatletter
\def\@seccntformat#1{\csname the#1\endcsname.\quad}
\def\numberline#1{\hb@xt@\@tempdima{#1\if&#1&\else.\fi\hfil}}
\makeatother
\renewcommand{\chapternumberline}[1]{#1.\quad}
\renewcommand{\cftchapterdotsep}{\cftdotsep}
\captionnamefont{\small}
\captiontitlefont{\small}
        \addto\captionspolish{
        \renewcommand{\tablename}{Tab.}
}


        \addto\captionspolish{
        \renewcommand{\figurename}{Rys.}
}


        \addto\captionspolish{
        \renewcommand{\bibname}{Bibliografia}
}

        \addto\captionspolish{
        \renewcommand{\listfigurename}{Spis rysunków} 
}

        \addto\captionspolish{
        \renewcommand{\listtablename}{Spis tabel}
}

        \addto\captionspolish{
\renewcommand\indexname{Indeks rzeczowy}
}
\addtopsmarks{headings}{
\nouppercaseheads
}{
\createmark{chapter}{both}{shownumber}{}{. \space}
\createmark{section}{right}{shownumber}{}{. \space}
}
\makeatletter
\copypagestyle{outer}{headings}
\makeoddhead{outer}{}{}{\small\itshape\rightmark}
\makeevenhead{outer}{\small\itshape\leftmark}{}{}
\makeoddfoot{outer}{\small\@author:~\@titleShort}{}{\small\thepage}
\makeevenfoot{outer}{\small\thepage}{}{\small\@author:~\@title}
\makeheadrule{outer}{\linewidth}{\normalrulethickness}
\makefootrule{outer}{\linewidth}{\normalrulethickness}{2pt}
\makeatother
\copypagestyle{plain}{headings}
\makeoddhead{plain}{}{}{}
\makeevenhead{plain}{}{}{}
\makeevenfoot{plain}{}{}{}
\makeoddfoot{plain}{}{}{}

\copypagestyle{empty}{headings}
\makeoddhead{empty}{}{}{}
\makeevenhead{empty}{}{}{}
\makeevenfoot{empty}{}{}{}
\makeoddfoot{empty}{}{}{}

\makeatletter
%Uczelnia
\newcommand\uczelnia[1]{\renewcommand\@uczelnia{#1}}
\newcommand\@uczelnia{}
%Wydział
\newcommand\wydzial[1]{\renewcommand\@wydzial{#1}}
\newcommand\@wydzial{}
%Kierunek
\newcommand\kierunek[1]{\renewcommand\@kierunek{#1}}
\newcommand\@kierunek{}
%Specjalność
\newcommand\specjalnosc[1]{\renewcommand\@specjalnosc{#1}}
\newcommand\@specjalnosc{}
%Tytuł po angielsku
\newcommand\titleEN[1]{\renewcommand\@titleEN{#1}}
\newcommand\@titleEN{}
%Tytuł krótki
\newcommand\titleShort[1]{\renewcommand\@titleShort{#1}}
\newcommand\@titleShort{}
%Promotor
\newcommand\promotor[1]{\renewcommand\@promotor{#1}}
\newcommand\@promotor{}
\def\maketitle{
  \pagestyle{empty}
	\fontfamily{\ebgaramond@family}\selectfont
	
\newlength{\tmpfboxrule}
\setlength{\tmpfboxrule}{\fboxrule}
\setlength{\fboxsep}{2mm}
\setlength{\fboxrule}{0mm} 
\setlength{\unitlength}{1mm}
\begin{picture}(0,0)
\put(26,-124){\fbox{
\parbox[c][71mm][c]{104mm}{\centering
\fontsize{16pt}{18pt}\selectfont \@title\\[5mm]
\fontsize{16pt}{18pt}\selectfont \@titleEN\\[20mm]
\fontsize{16pt}{18pt}\selectfont AUTOR:\\[2mm]
\fontsize{14pt}{16pt}\selectfont \@author}
}
}
\end{picture}
\setlength{\fboxrule}{\tmpfboxrule} 
	{\centering
		{\fontsize{22pt}{24pt}\selectfont \@uczelnia}\\[0.4cm]
		{\fontsize{22pt}{24pt}\selectfont \@wydzial}\\[0.5cm]
		  \hrule
	}
{\flushleft\fontsize{14pt}{16pt}\selectfont%
\begin{tabular}{ll}
KIERUNEK: & \@kierunek\\
SPECJALNOŚĆ: & \@specjalnosc\\
\end{tabular}\\[1.3cm]
}
{\centering
{\fontsize{32pt}{36pt}\selectfont PRACA DYPLOMOWA}\\[0.5cm]
{\fontsize{32pt}{36pt}\selectfont INŻYNIERSKA}\\[2.5cm]
}
\vfill
\begin{tabularx}{\linewidth}{p{6cm}X}
		&{\fontsize{16pt}{18pt}\selectfont PROWADZĄCY PRACĘ:}\\[2mm]
		&{\fontsize{14pt}{16pt}\selectfont \@promotor}\\[10mm]
		%&{\fontsize{16pt}{18pt}\selectfont OCENA PRACY:}\\[20mm]
	\end{tabularx}
\vspace{2cm}
\hrule\vspace*{0.3cm}
{\centering
{\fontsize{16pt}{18pt}\selectfont \@date}\\[0cm]
}
\normalfont
 \cleardoublepage
}
\makeatother

%%%%%%%%%%%%%%%%%%%%%%%%%%%%%%%%%%%%%%%%%
%%  Metadane dokumentu 
%%%%%%%%%%%%%%%%%%%%%%%%%%%%%%%%%%%%%%%%%
\title{Tworzenie mapy pomieszczenia za pomocą autonomicznej, mobilnej platformy skanującej}
\titleShort{Tworzenie mapy pomieszczenia za pomocą platformy skanującej}
\titleEN{Creating a room map using autonomous, mobile scanning platform}
\author{Tomasz Jakubowski}
\uczelnia{POLITECHNIKA WROCŁAWSKA}
\wydzial{WYDZIAŁ ELEKTRONIKI}
\kierunek{TELEINFORMATYKA}
\specjalnosc{PROJEKTOWANIE SIECI TELEINFORMATYCZNYCH}
\promotor{dr inż. Paweł Trajdos, K32W04D03}
\date{WROCŁAW, 2020}


\begin{document}
\pdfbookmark[0]{Tytuł}{Tytul.1}
\maketitle

\chapterstyle{noNumbered}
\pagestyle{outer}
\mbox{}\pdfbookmark[0]{Spis treści}{spisTresci.1}
\tableofcontents* 

% \newpage
% \mbox{}\pdfbookmark[0]{Spis rysunków}{spisRysunkow.1}
% %\addcontentsline{toc}{chapter}{Spis rysunków}
% \listoffigures*

% \newpage
% \mbox{}\pdfbookmark[0]{Spis listingów}{spisListingow.1}
% %\addcontentsline{toc}{chapter}{Spis listingów}
% \lstlistoflistings*

% \newpage
% \mbox{}\pdfbookmark[0]{Spis tabel}{spisTabel.1}
% %\addcontentsline{toc}{chapter}{Spis tabel}
% \listoftables*

\chapter*{Słownik skrótów i pojęć}\mbox{}\pdfbookmark[0]{Słownik skrótów i pojęć}{skroty.1}
\label{sec:skroty}
\noindent
\begin{description}[labelwidth=*]
  \item  [IoT] (ang. \emph{Internet of Things}) \emph{Internet Rzeczy} - koncepcja przedmiotów codziennego użytku samodzielnie komunikujących się za pośrednictwem sieci Internet bądź innego środka w celu gromadzenia i przetwarzania danych
  
  \item [ToF] (ang. \emph{Time of Flight}) \emph{Czas lotu} - tutaj w odniesieniu do rodzaju sensora laserowego mierzącego dystans za pomocą pomiaru czasu jaki upływa od wysłania impulsu świetlnego do odebrania jego odbicia od przeszkody
  
  \item [SLAM] (ang. \emph{Simultaneous Localization And Mapping}) \emph{Symultaniczna lokalizacja i mapowanie} - problem obliczeniowy dotyczący jednoczesnego tworzenia obrysu obiektów w otoczeniu na podstawie pomiarów i znanej lokalizacji jednostki mierzącej w przestrzeni oraz określenia położenia jednostki mierzącej w otoczeniu na podstawie dokonywanych przez nią pomiarów
  
  \item [VSLAM]
  
  \item [LIDAR]
  
  \item [USB-UART]
  
  \item [PC] (ang. \emph{Personal Computer}) \emph{Komputer Osobisty}
  
  \item [smart] (z ang. \emph{sprytny})  - w kontekście tego dokumentu jest to jedno z haseł często występujących w nazwach i reklamach produktów wyposażonych w dowolnego rodzaju mikrokomputer z oprogramowaniem pozwalającym na komunikację sprzętu z aplikacją na smartfona bądź interfejsem przeglądarkowym, rzekomo ułatwiającym korzystanie z niego. Ubarwia ono obraz urządzenia w oczach konsumenta mając przekonać go do jego innowacyjności. 
  
  \item [intelligent] (z ang. \emph{inteligentny})  - podobnie jak ``smart`` w kontekście niniejszej pracy jest często wykorzystywanym określeniem nowoczesnych produktów
  
  \item [Firmware] - oprogramowanie wbudowane działające na mikrokontrolerze
  
  \item [Arduino] - rodzaj płytek drukowanych z wbudowanymi mikrokontrolerami o publicznie dostępnych schematach elektronicznych, również otwartoźródłowe środowisko służące do pisania, kompilacji i wgrywania programów na płytki
  
  \item [STM32duino] - nieoficjalna implementacja środowiska kompilacji (bibliotek) Arduino dla płytek z mikrokontrolerami STM32
  
\end{description}

\chapterstyle{default}
\chapter{Wprowadzenie}
\section{Wstęp}\

Wraz z rozwojem technologii i wciąż rosnącym popytem na nowinki elektroniczne łatwo jest zaobserwować jak z biegiem czasu pewne rozwiązania adaptowane są do użytku codziennego. Sprzęt niegdyś będący jedynie drogą ciekawostką staje się akcesorium bez którego coraz trudniej jest wyobrazić sobie normalne życie. Tak było z komputerami osobistymi, płatnościami elektronicznymi, dostępem do Internetu, następnie do listy dołączyły smartfony, hulajnogi elektryczne czy douszne słuchawki bluetooth. Coraz popularniejsze stają się także rozwiązania IoT (ang. \emph{Internet of Things}) z zakresu AGD - inteligentne pralki, lodówki, odkurzacze. I to właśnie ostatnie z rozwiązań szczególnie przykuło uwagę autora. 

Podczas gdy wiele sprzętu reklamowanego z wykorzystaniem haseł takich jak ``smart`` czy ``intelligent`` z inteligencją nie ma zbyt wiele wspólnego, to ta granica zdaje się coraz mocniej zacierać - szczególnie w przypadku autonomicznych jednostek stworzonych w celu konserwacji powierzchni płaskich. I tutaj nasuwa się pytanie - Dlaczego? Otóż wiele z wymienionych wyżej urządzeń realizuje pojedyncze, wyznaczone, stosunkowo proste zadania a nowoczesnym dodatkiem do tego ma być łączność z siecią, zdalna obsługa z poziomu aplikacji czy duży, kolorowy wyświetlacz pokazujący aktualną pogodę. Oczywiście inteligentne odkurzacze również oferują podobne udogodnienia, jednak nie to wyróżnia je na tle innych akcesoriów. Tym czynnikiem jest złożoność, na pierwszy rzut oka prostego, zadania które realizują. Na drodze do budowy samodzielnie odkurzającego robota stoi wiele przeszkód związanych z ładowaniem, mapowaniem, nawigacją, omijaniem obiektów stojących na wyznaczonej trasie, radzeniem sobie ze zmieniającym się otoczeniem.

W tej pracy przedstawione zostanie autorskie podejście do jednego z tych problemów, przy którym zostaną wykorzystane zarówno własnoręcznie sporządzone jak i gotowe, publicznie dostępne rozwiązania. Tym problemem jest tworzenie mapy pokoju.

\section{Cel i zakres pracy}
\subsection{Cel}
Celem jest stworzenie działającego systemu składającego się z autonomicznego robota (platformy mobilnej) połączonego z aplikacją na komputerze stacjonarnym lub laptopie. Robot ten ma za zadanie samodzielnie poruszać się po pomieszczeniu i skanować je w poszukiwaniu przeszkód (ścian, foteli, nóg krzeseł, stołów itp.), a pozyskane dane przesyłać do komputera. W
aplikacji z danych zebranych z otoczenia robota ma powstać dwuwymiarowa mapa pomieszczenia opisująca je na poziomej płaszczyźnie ok. 10 cm nad poziomem podłogi.

\subsection{Zakres}
\begin{itemize}
    \item zaprojektowanie i zbudowanie jeżdżącego robota wyposażonego w laserowy sensor odległości i moduł Bluetooth do komunikacji z komputerem
    \item implementacja algorytmu pozwalającego na autonomiczne poruszanie
    \item napisanie aplikacji rysującej mapę pomieszczenia działającej na komputerze lub laptopie
\end{itemize}


\section{Koncepcja projektu}
Po pierwsze robot powinien posiadać zasilanie akumulatorowe - w przeciwnym wypadku wymagałby podania zasilania za pośrednictwem przewodu, co byłoby niepraktyczne. Aby zasilania starczyło na jak najwięcej czasu dobrze by było, aby nie wykorzystywał zgromadzonej energii na obliczenia, które mogą zostać wykonane po stronie aplikacji go kontrolującej. Z tego względu do kontroli peryferiów platformy został wykorzystany mikrokontroler.

Aby móc sporządzić mapę potrzebne będą lokalizacje przeszkód w postaci punktów reprezentowanych na dwuwymiarowej płaszczyźnie. Taki punkt można obliczyć znając odległość i kierunek do przeszkody względem robota - potrzebuje on więc  sensorów dzięki którym będzie w stanie tą odległość zmierzyć. Istnieją gotowe rozwiązania takie jak RPLidar A1M8 \cite{lidar-datasheet-6}, lub RPLIDAR A2M6 \cite{lidar-datasheet-10}, korzystające z lasera obracającego się na podstawie, jednak są na tyle kosztowne, że ostatecznie zdecydowano się na tańszą alternatywę - Lidar TF Luna. Różnica w cenie tańszego z dwóch wymienionych wcześniej czujników a wybranego to na dzień dzisiejszy ponad 1500 złotych. Sensor odległości zamontowany na wieżyczce obracanej za pomocą serwomechanizmu  w zakresie od 0° do 180° jest wystarczający do realizacji tego zadania.

Żeby pozycja wcześniej wspomnianych punktów była odwzorowywała rozkład pomieszczenia platforma powinna realizować zadanie nawigacji zliczeniowej (odometrii). Dzięki temu możliwe jest oszacowanie aktualnej pozycji i przemieszczenia robota w przestrzeni względem jego pozycji startowej. Dopiero korzystając z tych danych połączonych z kierunkiem i dystansem do przeszkody można umieścić ją na mapie.

Dla mniejszego poślizgu oraz łatwego pokonywania niewielkich, nieznaczących przeszkód takich jak przewody, listwy zasilające, dywaniki oraz dla ułatwienia obliczeń związanych z odometrią robot powinien być wyposażony w system napędowy o odpowiednio małym poślizgu.

Należy również ustalić w jakim języku programowania napisane będą programy - wszakże w zależności od środowiska programista powinien spodziewać się innych możliwości i ograniczeń. Ze względu na możliwość szybkiego prototypowania i czytelność kodu do napisania oprogramowania PC wybrany został język Python\cite{python}. Do zaprogramowania robota autor wybrał język C++\cite{cpp}. Do reprezentacji graficznej mapy stworzonej przez pojazd można posłużyć się środowiskiem \emph{Processing} \cite{GettingStartedWithProcessing} bądź podobną implementacją zgodną z językiem Python.

\section{Przegląd technologii}
Nie jest tajemnicą że inspiracją do tego projektu były coraz popularniejsze robotyczne odkurzacze. Warto jednak nadać pewien rys historii rozwoju tej branży i technologii które były stosowane na przełomie lat.

Pierwszym komercyjnie dostępnym robotem odkurzającym był Electrolux Trilobite\cite{vacuum-history}. Robot ten był wyposażony w sensory ultradźwiękowe dzięki czemu zachowywał odstęp od ścian. Sensory takie nie były wystarczające - ostre obiekty bądź takie o niewielkiej powierzchni mogły zostać przeoczone dlatego też dodatkowo miał on zderzak wciskający się podczas kontaktu z przeszkodą. Do tego wykrywał on uskoki takie jak np. schody za pomocą czujnika wykorzystującego promieniowanie podczerwone a także był w stanie omijać strefy ręcznie wyznaczone magnetycznymi paskami umieszczonymi na powierzchni czyszczonej. Z czasem również konkurencja zaczęła się wdrażać w rynek i oferować podobne rozwiązania.

Pierwsze odkurzacze nie korzystały z zaawansowanych algorytmów nawigacji, mając do dyspozycji jedynie ograniczony zasób informacji o otaczającym je środowisku ze względu na proste czujniki w nich montowane. Poruszały się one w sposób wręcz chaotyczny, ``odbijając`` się od przeszkód pod różnymi kątami, nie czyszcząc równomiernie całej powierzchni. Nie mniej jednak takie rozwiązanie było na dane czasy wystarczające i samo w sobie bardzo innowacyjne - zwalniało użytkownika z konieczności odkurzania ręcznego.

Z biegiem lat, rozwojem algorytmów nawigacji oraz (co najważniejsze w kontekście sprzedaży urządzeń na rynku konsumenckim) taniejącą elektroniką i coraz mniej kosztownymi sensorami zaczęto implementować bardziej złożone rozwiązania. Na dzień dzisiejszy inteligentne odkurzacze korzystają z dwóch technologii:
\begin{itemize}
\item SLAM - za pomocą laserowego sensora robot mierzy odległość od swojego punktu do przeszkód wokół. Zmierzone odległości jest w stanie przenieść na mapę. Mapa może być widoczna z poziomu aplikacji zarządzającej robotem i służyć np. do wyznaczania stref które ma omijać. Ponadto pozwoli mu odnaleźć się w danym domu i wyznaczyć sobie ścieżkę, obliczyć bieżący procent wykonanej pracy czy też wrócić do bazy ładującej
\item VSLAM (Visual SLAM) - w tym wypadku robot jest wyposażony w kamerę skierowaną bezpośrednio na sufit lub pod kątem. Dzięki zaawansowanym algorytmom przetwarzania obrazu wybiera on pewne punkty odniesienia na podstawie których szacuje swoją pozycję względem otoczenia. Technologia wymaga wyższej mocy obliczeniowej i jest mniej dokładna, nie mniej jednak pozostawia otwarte pole na innowacje w zakresie sztucznej inteligencji i rozpoznawania typu obiektów.
\end{itemize}

Ze względu na niewielkie doświadczenie autora w dziedzinie komputerowego przetwarzania obrazów, w niniejszej pracy wykorzystany zostanie jeden z algorytmów działających w technologii SLAM - Gmapping, oparty o filtr cząstek typu \emph{rao-blackwellized} \cite{Murphy2000}\cite{Grisetti2005}\cite{Grisetti2007}.

W Tab. \ref{tab:lidar-comparison} przedstawiono porównanie wybranych, dostępnych aktualnie na polskim rynku sensorów. Można wydzielić tutaj dwie zasadnicze grupy - jednostki mierzące punktowo oraz te które mierzą pewien obszar w jednym lub dwóch wymiarach. Te pozycje które nie mają określonego \emph{kąta widzenia w poziomie} należą do drugiej kategorii. Takie czujniki posiadają mechanizmy multipleksacji strumienia świetlnego, np. w postaci obrotowej wieżyczki na której zamontowany jest laser. Należy również zaznaczyć, że maksymalna częstotliwość pomiarów w tych sensorach oznacza pomiar całego mierzonego obszaru, w czujnikach mierzących punktowo dotyczy ona pojedynczego punktu.

Niestety, o ile jest to rozwiązanie wygodne, zwalniające inżyniera z potrzeby budowania własnej wieżyczki, o tyle takie sensory są dużo droższe - najtańszy z nich \emph{Slamtec RPLidar A1M8} kosztuje 500 złotych, będąc nawet poniżej półki cenowej droższych rozwiązań mierzących punktowo. Najtańszy z dostępnych \emph{Benewake Lidar TF Luna} posiada dwudziestocentymetrową martwą strefę, nie odstając znacząco od konkurencji, gdzie najmniejszą strefą charakteryzuje się ośmiokrotnie droższy sensor \emph{SparkFun Lidar Lite v3HP}. Martwą strefę można zmniejszyć, montując sensor w cofnięciu od osi obrotu, należy wówczas dodać odpowiedni dystans do zmierzonej wartości. Minusem takiego rozwiązania jest większy rozmiar wiązki przy takiej samej odległości od obiektu. 

Nie mniej jednak najniższy pobór energii, wystarczająca częstotliwość próbkowania i akceptowalny zasięg 8 metrów (przeciętny pokój ma wymiary od kilku do kilkunastu metrów mierząc od najodleglejszych punktów) są tutaj największymi atutami czujnika firmy \emph{Benewake}. Mając na uwadze wyżej wymienione parametry oraz ograniczony budżet autor zdecydował się na wykorzysytanie właśnie tego urządzenia.

\begin{table}
    \caption{Przegląd dostępnych na rynku sensorów LIDAR}
     \label{tab:lidar-comparison}
    \centering
    \begin{tabular}{ |p{0.1\linewidth}|p{0.1\linewidth}|p{0.1\linewidth}|p{0.12\linewidth}|p{0.12\linewidth}|p{0.08\linewidth}|p{0.08\linewidth}|p{0.05\linewidth}| } \hline
	Producent, model & Zasięg & Kąt widzenia w poziomie & Dokładność & Rozbieżność wiązki & Pobór energii & Maks. częst. pomiarów & Cena \\ \hline
	- & m & ° & m & ° & W & Hz & zł \\ \hline \hline
	Benewake Lidar TF Luna \cite{lidar-datasheet-1} & 0,2-8 & n/d & 0,06 (2\% dla >3m) & 2 & 0,35 & 100 & 89 \\ \hline
	Benewake Lidar TFMini-S \cite{lidar-datasheet-2} & 0,1-12 & n/d & 0,06 (1\% dla >6m) & 2 & 0,7 & 100 & 180 \\ \hline
	Benewake Lidar TFMini Plus \cite{lidar-datasheet-3} & 0,1-12 & n/d & 0,05 (1\% dla >6m) & 3,6 & 0,55 & 1000 & 200 \\ \hline
	Benewake Lidar TF02 Pro \cite{lidar-datasheet-4} & 0,1-40 & n/d & 0,05 (1\% dla >5m) & 3 & 1 & 100 & 400 \\ \hline
	Benewake Lidar TF02 \cite{lidar-datasheet-5} & 0,4-22 & n/d & 0,06 (2\% dla >5m) & 3 & 1 & 100 & 490 \\ \hline
	Slamtec RPLidar A1M8 \cite{lidar-datasheet-6} & 0,15-12 & 360 & 0,0005 (1\% dla >1,5m) & 1 & 2 & 10 & 500 \\ \hline
	SparkFun Lidar Lite v3 \cite{lidar-datasheet-7} & 0-40 & n/d & 0,025 & 0,46 & 0,65 & 500 & 760 \\ \hline
	SparkFun Lidar Lite v3HP \cite{lidar-datasheet-8} & 0,05-40 & n/d & 0,025 & 0,46 & 0,43 & 1000 & 800 \\ \hline
	Slamtec RPLidar A2M8 \cite{lidar-datasheet-9} & 0,15-8 & 360 & 0,0005 (1\% dla >1,5m) & 0,9 & 2,25 & 15 & 1600 \\ \hline
	Slamtec RPLidar A2M6 \cite{lidar-datasheet-10} & 0,15-18 & 360 & 0,0005 (1\% dla >1,5m) & 0,9 & 2,25 & 15 & 2900 \\ \hline
	Benewake Lidar CE30-A \cite{lidar-datasheet-11} & 0,1 - 4 & 130 & 0.06 & 0,41 & 6 & 20 & 3700 \\ \hline
\end{tabular}
\end{table}











\chapter{Projekt}
\section{Oprogramowanie PC}
\subsection{Struktura oprogramowania}
Robot wykonuje polecenia zadane przez aplikację sterującą oraz odpowiada na zapytania o dane. Aplikacja w ostatecznej wersji została zaimplementowana jako moduł platformy ROS\cite{ros} na systemie Xubuntu 20.04. Po jej uruchomieniu ukazują się dwa okna - jedno jest oknem głównym do kontroli pojazdu, widoczne na Rys. \ref{fig:app-main-window}. Drugie to okno programu RViz w którym widoczny jest podgląd pozycji robota oraz mapy którą sporządza ukazane na Rys. \ref{fig:rviz-window}.

\subsubsection{Okno główne programu}
\begin{figure}[ht]
	\centering
		\includegraphics[width=1\linewidth]{rys/main-app-view-3.PNG}
	\caption{Okno główne aplikacji sterującej}
	\label{fig:app-main-window}
\end{figure}

Okno główne zostało podzielone na segmenty oraz zakładki w celu odseparowania poszczególnych funkcjonalności. Opis poszczególnych sekcji:

Sekcja \emph{CONNECTION} dotyczy połączenia portu szeregowego i zawiera elementy takie jak:
\begin{itemize}
    \item Lista rozwijana do wyboru portu szeregowego do którego wpięty jest konwerter USB-UART
    \item Przycisk \emph{refresh} odświeża listę
    \item Przycisk \emph{connect} inicjuje połączenie
    \item Ikona statusu połączenia - krzyżyk oznacza jego brak, zegarek oznacza oczekiwanie a symbol fajki oznacza zainicjowane połączenie
\end{itemize}

Sekcja \emph{RAW CONTROL} zawiera kontrolki sterujące robotem. Przytrzymanie każdego z przycisków odpowiada ruchowi robota w przód i w tył (kolejno strzałki w górę i w dół) lub obrotowi w lewo i w prawo (strzałki w lewo i w prawo). Po puszczeniu przycisku platforma zatrzymuje się bez zwłoki. Suwak służy do ustawiania pozycji wieżyczki pomiarowej, jest wyskalowany w stopniach.
\\

Sekcja \emph{ODOMETRIC CONTROL} służy do jazdy odometrycznej, tj. poruszając się za pomocą tych kontrolek uwzględniany jest przejechana trasa. Analogicznie jak w poprzedniej sekcji poruszanie kontrolowane jest przez strzałki, aczkolwiek tutaj nie należy ich przytrzymywać a jedynie krótko kliknąć. Po kliknięciu zostanie pokonany pewien dystans po którym robot się zatrzyma. Dystans do przejechania (krok) ustawiany jest za pomocą suwaka \emph{drive step} a wartość przez niego reprezentowana jest podana w centymetrach. Aby ustawić krok obrotu należy skorzystać z suwaka \emph{rotate step}. Przycisk \emph{ROTATE TO} służy do obracania robota na wyznaczony azymut, którego wartość wcześniej należy ustawić suwakiem znajdującym się obok niego. (Uwaga! Pojęcie azymutu w tym dokumencie jest użyte w odniesieniu do zmiennej trzymającej aktualny kąt skierowania robota, nie azymtu geograficznego. Więcej o tym znajduję się w rozdziale >TODO dac refa to miejsca gdzie to jest obszernie wytlumaczone)
\\

Sekcja \emph{CONSOLE} pozwala na ręczne sterowanie robotem za pośrednictwem komend. Tutaj należy zaznaczyć że wszysktie funkcje sterujące oferowane przez program korzystają z tych komend odpowiednio je formułując i wysyłając do jednostki mobilnej. Pole tekstowe podpisane \emph{command line} służy do wprowadzania komendy, która jest wysyłana po kliknięciu przycisku \emph{SEND}.
Owa komenda pojawi się wtedy w polu tekstowym poniżej oznaczonym \emph{output}. Przycisk \emph{CLEAR BUFFER} wyczyści okno podglądu wraz z buforem nadawczym i odbiorczym, co przydaje się przy (na szczęście bardzo rzadkich) problemach z zerwaną komunikacją lub utratą synchronizacji sekwencji przesyłanych komend i odpowiedzi.
\\

Sekcja \emph{BASIC FUNCTIONS} posiada kolekcję przycisków wykonujących podstawowe funkcje robota niezwiązane z poruszaniem.
\begin{itemize}
    \item \emph{BEEP} powoduje wydanie 3 krótkich dźwięków przez robota
    \item \emph{PRINT DUCK} rysuje małą kaczkę na ekranie
    \item \emph{GET\_DISTANCE} mierzy i zwraca odległość od robota do przeszkody na którą aktualnie wycelowana jest wieżyczka
    \item \emph{GET\_AZIMUTH} mierzy i zwraca azymut w którym skierowany jest robot
    \item \emph{GET\_MAG} zwraca surowe dane z magnetometru (osie X i Y)
    \item \emph{GET\_MAG\_CAL} zwraca dane kalibracji magnetometru
    \item \emph{SCAN} skanuje otoczenie (obraca wieżyczkę i dokonuje serii 180 pomiarów odległości), po czym zwraca zmierzone wartości
    \item \emph{KILL} odłącza sygnał sterujący od serwomechanizmów
\end{itemize}

Sekcja \emph{EXTENDED FUNCTIONS} pozwala na wykonywanie bardziej złożonych zadań
\begin{itemize}
    \item \emph{AUTO DRIVE} załącza algorytm autonomiczej jazdy robota
    \item \emph{GET\_AZIMUTH USING KALMAN} wykonuje serię pomiarów i za pomocą prostej implementacji filtra Kalmana zwraca przefiltrowany wynik reprezentujący azymut w którym robot jest skierowany
\end{itemize}

Zakładka \emph{map} zawiera wykres który służy do wyświetlania zmierzonych punktów na układzie współrzędnych

Zakładka \emph{raw output} zawiera pełnowymiarowy podgląd na historię wysyłanych i odbieranych danych. Jest większą wersją okienka \emph{output} sekcji \emph{CONSOLE}

Zakładka \emph{magnetometer calibration} zawiera zestaw elementów wykorzystywanych do kalibracji magnetometru. Więcej o tej zakładce jak i o przebiegu kalibracji w podrozdziale \ref{sec:mag_cal}.

\subsubsection{Program RViz}
\begin{figure}[ht]
	\centering
		\includegraphics[width=1\linewidth]{rys/main-app-view-4.PNG}
	\caption{Okno programu RViz}
	\label{fig:rviz-window}
\end{figure}

Z okna głównego programu RViz należy zwrócić uwagę jedynie na podgląd układu współrzędnych po prawej stronie okna. Wszystkie parametry są ustawiane automatycznie wraz ze startem aplikacji, nie ma potrzeby żadnej ręcznej konfiguracji. Więcej informacji o tym programie dostępne jest na stronie internetowej \cite{rviz}.

\subsubsection{Środowisko sterujące}
\label{sec:ros-env}
\begin{figure}[ht]
	\centering
		\includegraphics[width=1\linewidth]{rys/pc-application-infrastructure.png}
	\caption{Struktura środowiska ROS}
	\label{fig:pc-app-ros-infrastructure}
\end{figure}

Aplikacja komputerowa po uruchomieniu w środowisku ROS widoczna jest jako jeden z węzłów (node). W celu sporządzania mapy komunikuje się z innymi węzłami za pośrednictwem tematów (topics).
Na Rys. \ref{fig:pc-app-ros-infrastructure} ukazany jest schemat komunikacji poszczególnych węzłów.
Wykorzystywane są trzy tematy:
\begin{itemize}
    \item \emph{tf} jest tematem na którym publikowane są wszelkie translacje, czyli zależności między układami odniesienia
    \item \emph{scan} - tutaj publikowane są wyniki pomiarów odległości po wykonaniu skanu przez robota
    \item \emph{map} - na tym temacie pojawiają się dane reprezentujące mapę, wygenerowane przez algorytm \emph{gmapping}\cite{Grisetti2005}\cite{gmapping-website}\cite{gmapping-ros}
\end{itemize}

Ramki (frames), czyli w terminologii stosowanej w ROS układy odniesienia, a konkretnie ich względna pozycja są reprezentowane za pomocą wektorów zawierających dane o przesunięciu jak i obrocie. Więcej informacji na temat szczegółowego działania ramek można zasięgnąć na stronie internetowej ROS'a\cite{ros}. W niniejszym projekcie wykorzystywane są 4 ramki:
\begin{itemize}
    \item \emph{map} to główny układ odniesienia względem którego rysowana jest mapa
    \item \emph{odom} jest punktem odniesienia względem którego działa algorytm nawigacji zliczeniowej
    \item \emph{base_link} odnosi się do bazy pojazdu, tj. środka względem którego się obraca podczas zakręcania
    \item \emph{base_laser} to pozycja sensora skanującego
\end{itemize}
Wsyzstkie translacje między ramkami publikowane są na temacie \emph{tf}.

Węzeł \emph{laser\_static\_pod\_broadcaster} cyklicznie publikuje informację o położeniu modułu skanującego względem bazy robota. Jako że znajduje się ona bezpośrednio nad środkiem obrotu platformy, a sporządzana mapa jest dwuwymiarowa, jest to wektor zerowy.

Węzeł \emph{scanbot\_communicator} publikuje swoją pozycję na temacie \emph{tf} oraz podczas wykonywania skanu na kanale \emph{scan}.

Węzeł \emph{slam\_gmapping} subskrybuje kanały \emph{tf} i \emph{scan}. Na ich podstawie odpowiednio filtrując punkty uzyskane z pomiarów i dopasowując je do poprzednich algorytm tworzy mapę przeszkód widzianych przez robota.

Węzeł \emph{rviz} reprezentuje program RViz i subskrybuje dwa tematy - \emph{tf} oraz \emph{map}. Pozycje układów odniesienia reprezentowane są prze trójkolorowe trójwymiarowe obiekty, gdzie każdy z kolorów odpowiada jednej z osi (czerwony, zielony, niebieski odpowiadają kolejno osiom X, Y oraz Z). Ich położenie jest znane dzięki subskrypcji pierwszego z wymienionych tematów. Za pomocą danych przychodzących z drugiego z nich, program rysuje mapę pomieszczenia.


\subsubsection{Struktura aplikacji sterującej}
\begin{figure}[ht]
	\centering
		\includegraphics[width=0.8\linewidth]{rys/pc-application-simplified-uml.png}
	\caption{Uproszczony diagram kluczowych klas aplikacji}
	\label{fig:simple-class-diagram}
\end{figure}

W tej sekcji przedstawiony zostanie zarys najbardziej znaczących klas aplikacji sterującej. Jako że sama jej budowa nie jest kluczowym aspektem niniejszej pracy, dlatego po szczegółowe informacje dotyczące jej funkcjonowania autor odsyła do kodu źródłowego znajdującego się w dodatku A.

Na diagramie \ref{fig:simple-class-diagram} ukazano najważniejsze klasy głównego programu. Głównym obiektem jest tutaj okno, instancja klasy \emph{MainWindow}. Tutaj obsłużone jest kreowanie interfejsu graficznego, czyli ustawienie przycisków, pól tekstowych i innych elementów w odpowiednich miejscach, nadanie im identyfikatorów, ustawienie ciągów tekstowych, suwaków i innych elementów graficznych. Również akcje towarzyszące kliknięciom przycisków i przesuwaniu suwaków są tutaj przypisywane do odpowiednich funkcji. Funkcje te dla lepszej organizacji oraz czytelności zostały przeniesione do innych klas zorientowanych wokół konkretnych segmentów pracy aplikacji. Obiekty tych klas dostępne są poprzez zmienne wewnątrz głównej klasy.
\\

Klasa \emph{MagCalPlot} obsługuje szereg funkcji związanych z kalibracją magnetometru \ref{sec:mag_cal}. Zawiera przyciski uruchamiające funkcje związane z:
\begin{itemize}
    \item Czyszczeniem wykresu
    \item Rysowaniem zmierzonych danych w postaci punktów na wykresie
    \item Automatycznym pomiarem surowych danych z magnetometru
    \item Ręcznym pomiarem surowych danych z magnetometru
    \item Korekcją błędu \emph{hard iron}
    \item Korekcją błędu \emph{soft iron}
    \item Wysyłaniem danych kalibracyjnych do robota
\end{itemize}

Klasa \emph{PolePlots} obsługuje rysowanie punktów oraz czyszczenie wykresu zakładki \emph{map}. Przechowuje również dane o kolorach punktów jakie są na nim rysowane.

Klasa \emph{Robot} odzwierciedla surowy interfejs robota \label{sec:firmware}, dodając warstwę abstrakcji na surowe komendy wydawane mobilnej platformie i rozszerzając je - przykładowo podczas każdego z wywołanych funkcją pomiarów azymutu, jest on od razu publikowany na temacie \emph{tf}. Dodatkowo, w tej klasie obsłużony jest pomiar azymutu za pośrednictwem filtra Kalmana \cite{Kedzierski2016}. Również tutaj obsłużone jest publikowanie danych z odometrii i wykonywanych skanów na odpowiednich tematach, opisanych wcześniej w sekcji \ref{sec:ros-env} jak i również algorytm samodzielnej jazdy robota.

Klasa \emph{SerialConnection} służy do obsługi komunikacji interfejsu szeregowego konwertera USB-UART. Zawiera funkcje służące do nawiązywania połączenia, wysyłania komend i odbierania odpowiedzi od robota. Ponadto posiada uchwyt do klasy głównej, dzięki czemu wymieniane dane prezentuje zarówno w polu tekstowym w sekcji okna \emph{CONSOLE} jak i w zakładce \emph{raw output}.

\subsection{Jazda autonomiczna}
W celu realizacji samodzielnej jazdy robot został wyposażony w autorski algorytm pozwalający na omijanie napotkanych przeszkód i poruszanie się wzdłuż ścian. Operuje on na bardzo prostych zasadach - obot wykonuje kroki, pomiędzy którymi skanuje otoczenie i decyduje o podjętej akcji. Decyzja zapada na podstawie wykrycia przeszkód w wyznaczonych strefach. Ze względu na korelację z napisanym programem Rys. \ref{fig:autodrive-zones} przedstawia angielskie nazwy tychże stref:
\begin{itemize}
    \item \emph{FRONT ZONE} (strefa F)
    \item \emph{LEFT ZONE 1} (strefa L1)
    \item \emph{LEFT ZONE 2} (strefa L2)
\end{itemize}

\begin{figure}[ht]
	\centering
		\includegraphics[width=1\linewidth]{rys/autodrive-zones.png}
	\caption{Strefy algorytmu jazdy autonomicznej}
	\label{fig:autodrive-zones}
\end{figure}

Należy zwrócić uwagę, że  Rys. \ref{fig:autodrive-zones} ma charakter poglądowy i nie jest wykonany w skali. Platforma znajduje się tutaj w punkcie środkowym układu współrzędnych, rysunek przedstawia zakres kątów od 0 do 180 stopni, liczonych od kierunku prawego przeciwnie do ruchu wskazówek zegara. W tej samej kolejności przedstawiane są dane zeskanowane przez robota. Rozmiary stref zostały zdefiniowane w kodzie źródłowym programu i empirycznie dostosowane do stanu w którym pozwalają osiągnąć zadowalający poziom pracy algorytmu.
Wielkości kroków to kolejne z parametrów którymi można regulować jakość pracy algorytmu. Rys. \ref{fig:autodrive-algorithm} prezentuje jego funkcjonowanie w oparciu o wymienione wartości.


\begin{itemize}
    \item zakres kątów obejmujących strefę \emph{F} to <70, 110>
    \item zakres kątów obejmujących strefy \emph{L1} i \emph{L2} to <140, 180>
    \item promień strefy \emph{F} wynosi 30 cm
    \item promień strefy \emph{L1} wynosi 50 cm
    \item strefa \emph{L2} jest wycinkiem zaczynającym się od promienia równego 70 cm, kończąca się na 90 cm
    \item mały krok jazdy do przodu wynosi 10 cm
    \item mały krok obrotu wynosi 10 stopni
    \item duży krok obrotu wynosi 30 stopni
\end{itemize}



\begin{figure}[ht]
	\centering
		\includegraphics[width=1\linewidth]{rys/autodrive-algorithm.png}
	\caption{Algorytm jazdy autonomicznej}
	\label{fig:autodrive-algorithm}
\end{figure}






\subsection{Skan otoczenia}
>TODO tu napisac o bazowym mapowaniu pkt na plaszczyzne
o swojej implementacji ze scorem
nastepnie o tym co moznaby dalej (translacja korekta)
ale nie zostalo zrobione bo bylo slabe i to wymaga filtrow czasteczek
i ze jest cos takiego jak cartographer od googla
ale zdecydowano ze gmapper jest spoko i czemu nie gmapper

\subsection{Odometria}
\subsubsection{Kalibracja magnetometru}
\label{sec:mag_cal}
>TODO koniecznie daj screenshoty
opisz hard i soft iron offset
powiedz o filtrze kalmana

\subsubsection{UMBenchmark\cite{Borenstein1995}}
>TODO tu dac fotki koniecznie
jakies rysunki nabazgrane z kwadratami
i powiedziec ze sie nie pokrywalo najlepiej
dac tabelke z pozycjami
pokazac obliczenia center of gravity itd
ej wyszlo ~1% czyli to bez sensu
robot juz jezdzi dobrze
to zasługa wielkiego myśliciela autora tego tekstu
oraz komitetu centralnego
a jakże

\section{Firmware}
\label{sec:firmware}
>TODO powiedziec o ograniczeniach wynikajacych ze stosowania
mikrokontrolera, o srodowisku stm z ktorego nie skorzystalem
o srodowisku stm32duino o tym ze stm32 jest szybsze i wgl
super bo to wszystko zre mniej pradu niz rasp i nie ma problemu kiedy zasilanie spada

\section{Schemat budowy mechanicznej}
>TODO powiedz co sie zmienialo, jaki jest problem z gasienicami
wyjasnij ze gasienice upraszczamy do 2 kolek

\section{Schemat elektroniczny}
>TODO wspomnij o sensorach ultradzw oraz sharp IR



% to trzeba wrzucic gdzies moze do opisu widoku okna rviz?
% Aktualną pozycję można zaobserwować w drugim oknie, w widoku mapy jako obiekt z etykietą \emph{base_link} - podczas jazdy jego pozycja jest aktualizowana względem etykiety \emph{odom}.
\chapter{Ewaluacja}

\subsection{Odometria}
\label{sec:odometry}
Robot korzysta z nawigacji zliczeniowej w celu połączenia pomiarów odległości zebranych z otoczenia i aproksymacji ich położenia na płaszczyźnie względem jednego ustalonego punktu. Zliczaniu podlegają dwie wartości - dystans przejechany wzdłuż oraz obrót pojazdu w miejscu. Szereg zmierzonych w ten sposób wartości umożliwia odtworzenie przejechanej ścieżki. Rys. \ref{fig:odom-axis-simplified} przedstawia rzeczywisty oraz uproszczony model podwozia robota. Zamiast uwzględniać obie poziome osi i 4 koła, upraszcza się go do formy robota o jednej osi, z jedną parą kół. Gdy oba koła poruszają się w tę samą stronę, robot przesuwa się w przód lub w tył. Gdy pracują przeciwbieżnie - obraca się wokół punktu oznaczonego znakiem X.

\begin{figure}[H]
	\centering
		\includegraphics[width=0.6\linewidth]{rys/robot-odometry-simplified.pdf}
	\caption{Model rzeczywisty i uproszczony (X oznacza środek robota - pionową oś wokół której się obraca)}
	\label{fig:odom-axis-simplified}
\end{figure}


Pierwszym pomysłem było wykorzystanie enkoderów do pomiaru translacji pojazdu wzdłuż jego osi ruchu, pozostawiając zliczanie obrotu funkcjom korzystającym z magnetometru.

\begin{figure}[H]
	\centering
		\includegraphics[width=0.5\linewidth]{rys/encoder-position.png}
	\caption{Umiejscowienie enkodera z prawej strony robota}
	\label{fig:encoder-pos}
\end{figure}

Enkodery zostały zamontowane w miejscu przedstawionym na Rys. \ref{fig:encoder-pos}, symetrycznie po obu stronach podwozia. Napęd podany od serwa przez zębatkę przenosi napęd zarówno na tylne koło jak i enkoder, z przekładniami 1:1 w obu przypadkach. Dzięki takiemu rozwiązaniu pełny obrót enkodera odpowiada pełnemu obrotowi koła. Enkoder obrotowy EC-11 generuje 20 impulsów przy kącie obrotu 360°. Aby można było wyznaczyć pokonaną odległość, w pierwszej kolejnosci należy obliczyć stosunek ilości impulsów do przejechanego dystansu - w tym celu autor stworzył prosty wzór:

\begin{center}
    $DPR = \frac{2 \pi r}{p}$ \\
    \emph{DPR - współczynnik (ang. Distance to Pulse Ratio) wyrażany w cm/impuls} \\
    \emph{r - promień koła pojazdu} \\
    \emph{p - liczba impulsów enkodera na obrót koła}
\end{center}

Wiedząc ile impulsów generuje enkoder, oraz znając wymiary koła w łatwy sposób można obliczyć DPR. Podstawiając dane to jest:

\begin{center}
    $DPR = \frac{2 \pi 2,5cm}{20imp}$ \\
    $DPR = 0.785 \frac{cm}{imp}$ \\
    $\frac{1}{DPR} = 1.274$
\end{center}

Znając współczynnik aby obliczyć przejechany dystans wystarczy zmierzyć ilość impulsów jakie wystąpiły podczas przejazdu a następnie pomnmożyć je przez DPR. Otrzymany wynik oznacza przesunięcie robota wyrażone w centymetrach. Jako że platforma posiada dwa enkodery, a nie jest możliwe zachowanie idealnie prostego toru jazdy, wyciągana jest średnia liczba impulsów. Ze względu na grubość i wypustki na gąsienicach oraz ich poślizg rzeczywisty dystans przejechany będzie inny od zadanego. Dla kompensacji tej różnicy wartość DPR została skorygowana ręcznie do takiej, przy której błąd przemieszczenia robota zawierał się w granicy ±10\% na zadanym dystansie 100cm. Widoczna w kodzie programu, skorygowana wartość $\frac{1}{DPR}$ wynosi $1.325$ co odpowiada $DPR=0.755$ . Poniższy listing prezentuje funkcję realizującą to zadanie.


\begin{lstlisting}[basicstyle=\footnotesize\ttfamily,language=c++,caption=Fragment kodu obsługującego polecenie \emph{MOVE},label=lst:move]
//MOVE
case 9:
{
    reset_encoders();
    float val = getArgument(command, 1).toInt() * 1.325;

    if (val > 0)
        driveMotors(90, 90);
    else
        driveMotors(-90, -90);

    val = abs(val);
    while ((left_encoder_counter + right_encoder_counter) / 2 < val)
    {
    }

    driveMotors(0, 0);
    delay(100);
    Serial2.println("OK");
}
\end{lstlisting}

Podczas wykonywania polecenia \emph{MOVE} na gąsienice zadawana jest pełna prędkość i w pętli sprawdzana jest średnia ilość impulsów wygenerowanych przez enkoder. Dla optymalizacji algorytmu, zamiast przeliczać przy każdym pomiarze ilość impulsów razy wartość DPR, na początku zadana w parametrze wartość odległości jest mnożona przez $\frac{1}{DPR}$, i dalej w takiej formie ta wartość wykorzystywana przy operacji porównania. W momencie w którym średnia zliczona ilość impulsów przekroczy jej wartość, serwa zostają zatrzymane.
\\ 

Obrót (funkcja \emph{ROTATE}) polega na zadaniu przeciwstawnych wartości prędkości na obie gąsienice. W celu obrotu poruszają się one przeciwbieżnie, z równymi prędkościami. Pierwsza implementacja funkcji obrotu wyglądała jak przedstawiono poniżej:

\begin{lstlisting}[basicstyle=\footnotesize\ttfamily,language=c++,caption=Funkcja będąca głównym elementem obsługi poleceń \emph{ROTATE} oraz \emph{ROTATE\_TO},label=lst:rotate-to-function]
void rotateTo(int azimuth)
{
    if (calcAngleDistance(getAzimuth(), azimuth) > 0)
        driveMotors(-90, 90);
    else
        driveMotors(90, -90);

    while (true)
    {
        if (abs(calcAngleDistance(getAzimuth(), azimuth)) <= 10)
            break;
        delay(10);
    }
    driveMotors(0, 0);
    delay(100);
}
\end{lstlisting}

Na początku mierzony jest azymut początkowy i obliczany jest azymut końcowy (tj. początkowy + zadana wartość obrotu). Dalej wykonywana jest ta sama funkcja co w przypadku \emph{ROTATE\_TO} przyjmująca parametr azymutu końcowego. Podczas obrotu, w pętli, sprawdzana jest różnica kąta aktualnego od zadanego. Jeżeli wartość bezwzględna obrotu znajdzie się w zakresie ±10° robot zatrzyma się.

Takie rozwiązanie jest dobre, o ile magnetometr jest skalibrowany i funkcjonuje poprawnie. Niestety, podczas skanów okazało się że nie można polegać na pomiarach z tego sensora - więcej o tym znajduje się w sekcji \ref{sec:scan}. Z tego powodu koniecznym okazała się zmiana podejścia do pomiaru obrotu.

Poniższy listing przedstawia nowe podejście osbługi polecenia \emph{ROTATE}. Jest ono mniej dokładne od poprzedniego, bardziej podatne na dryf (błąd obrotu kumuluje się), natomiast taki pomiar jest odporny na zakłócenia pola magnetycznego.
Tym razem procedura jest analogiczna jak w przypadku ewaluacji polecenia \emph{MOVE}. W pętli sprawdzana jest średnia ilość impulsów, jednak tym razem gąsienice poruszają się przeciwbieżnie. Kąt obrotu na początku należy przemnożyć przez pewien współczynnik tak aby odpowiadał on ilości impulsów. Platforma kończy ruch w momencie gdy średnia wartość przekroczy obliczony próg impulsów.

\begin{lstlisting}[basicstyle=\footnotesize\ttfamily,language=c++,caption=Nowa implementacja obsługi polecenia \emph{ROTATE},label=lst:rotate]
// ROTATE
    case 11:
    {
        reset_encoders();
        float val = getArgument(command, 1).toInt() * 0.164;
        if (val > 0)
            driveMotors(-90, 90);
        else
            driveMotors(90, -90);

        val = abs(val);
        while ((left_encoder_counter + right_encoder_counter) / 2 < val)
        {
        }

        delay(100);
        driveMotors(0, 0);
        Serial2.println("OK");
    }
    break;
\end{lstlisting}

Stosunek kąta obrotu do liczby impulsów ($0.164$) został wyznaczony eksperymentalnie. Najpierw przyjęto wartość 1. Następnie zadawany był kąt obrotu 90°, po czym mierzony był rzeczywisty kąt obrotu. Jeżeli wynosił on więcej, współczynnik redukowano o 0,1; analogicznie gdy obrót był mniejszy niż zadano. Gdy tak zgrubna regulacja była niewystarczająca (albo za mały albo za duży obrót), krok został zmniejszony do 0,05. Po kolejnej zmianie kroku do 0,02 czynność powtarzano do momentu w którym dalsza korekcja nie była konieczna. \\

Taka implementacja została zachowana do końca projektu. Polecenie \emph{ROTATE\_TO} dalej korzysta ze starego sposobu z użyciem magnetometru, jednak nie jest ono wykorzystywane podczas pomiarów.

\subsubsection{Kalibracja magnetometru}
Wykorzystywany w projekcie moduł magnetometru jest urządzeniem czułym i bardzo podatnym na zakłócenia - szczególnie jeśli mierzone jest pole magnetyczne ziemi. Poza zakłóceniami z otoczenia, występują również te wynikające z samej budowy robota - wszelkie przewody przez które płynie prąd generują swoje własne pola, nawet same metalowe elementy interferują z pomiarem. >TODO przypis o zakloceniach i ich zrodlach Na te zakłócenia należy zwrócić uwagę i podjąć kroki mające na celu ich kompensację. Warto też wspomnieć, że w tej pracy jednostki uzyskane z pomiaru nie mają znaczenia - jedyne co jest potrzebne w celu uzyskania informacji o kierunku w którym robot jest zwrócony to kierunek i zwrot zmierzonego wektora wartości natężenia pola.

Pierwszym problemem są zakłócenia typu \emph{hard iron}. Ich obecność przejawia się w postaci stałego przesunięcia mierzonych we wszystkich trzech osiach wartości natężenia pola magnetycznego. Zmierzone wartości można przedstawić na trójwymiarowym wykresie (jak uczyniono w pierwszych wersjach aplikacji sterującej). Każda z osi wykresu odpowiada osi pomiaru natężenia pola magnetycznego. W idealnej sytuacji zbór punktów powinien być osadzony na sferze o środku w punkcie $(0,0,0)$. Tak się jednak nie dzieje, co jest widoczne na Rys. \ref{fig:3d-mag-no-cal}.

\begin{figure}[ht]
	\centering
		\includegraphics[width=0.6\linewidth]{rys/ScanBot-03-magnetometer-3d-decalibrated.PNG}
	\caption{Wykres obrazujący dane pozyskane z nieskalibrowanego magnetometru}
	\label{fig:3d-mag-no-cal}
\end{figure}

Aby dokonać korekcji \emph{hard iron} \cite{hard-iron}\cite{hard-soft-iron} dla każdej osi z osobna należy wyznaczyć minimalną i maksymalną zmierzoną wartość. Sumę obu wartości dzieli się przez 2, a uzyskana liczba to przesunięcie (ang. \emph{offset}). Aplikacja korekcji polega na odjęciu tej liczby od zmierzonej wartości.
Tą procedurę dla trzech osi przedstawiają kolejno Rys. \ref{fig:3d-mag-hard-corr-x}, Rys. \ref{fig:3d-mag-hard-corr-y} i Rys. \ref{fig:3d-mag-hard-corr-z}.

\begin{figure}[H]
	\centering
		\includegraphics[width=0.6\linewidth]{rys/ScanBot-04-magnetometer-3d-calibration.PNG}
	\caption{Korekta przesunięcia \emph{hard iron} dla osi X}
	\label{fig:3d-mag-hard-corr-x}
\end{figure}

\begin{figure}[H]
	\centering
		\includegraphics[width=0.6\linewidth]{rys/ScanBot-05-magnetometer-3d-calibration.PNG}
	\caption{Korekta przesunięcia \emph{hard iron} dla osi Y}
	\label{fig:3d-mag-hard-corr-y}
\end{figure}

\begin{figure}[H]
	\centering
		\includegraphics[width=0.6\linewidth]{rys/ScanBot-06-magnetometer-3d-calibration.PNG}
	\caption{Korekta przesunięcia \emph{hard iron} dla osi Z}
	\label{fig:3d-mag-hard-corr-z}
\end{figure}

Niestety, kształt chmury punktów wciąż daleki jest od idealnej sfery. Efektem tego jest nieliniowość uzyskanego azymutu - zmiana rzeczywistego kąta skierowania platformy względem zmiany zmierzonej będzie się znacząco różniła w zależności od początkowej pozycji. Ten efekt można by obejść za pomocą mapowania wartości przepuszczając odczyt przez funkcję odpowiedniej krzywej, jednak wymagałoby to formułowania skomplikowanego równania, bądź wyznaczenia arbitralnego przebiegu funkcji. Istnieje jednak mniej wymagające obliczeniowo podejście z zastosowaniem macierzy \cite{hard-soft-iron}. Dla uproszczenia obliczeń, zrezygnowano z pomiaru w trzech osiach, zamiast tego mierzone są jedynie natężenia pola w osiach X i Y (Rys. \ref{fig:2d-mag-no-cal}).

\begin{figure}[H]
	\centering
		\includegraphics[width=0.8\linewidth]{rys/ScanBot-08-2d-calibration-theta-sigma.PNG}
	\caption{Rozkalibrowany magnetometr na dwuwymiarowej płaszczyźnie}
	\label{fig:2d-mag-no-cal}
\end{figure}

Tak jak w przypadku trzech wymiarów, w pierwszej kolejności dokonywana jest korekcja zniekształceń \emph{hard iron} na wszystkich osiach (Rys. \ref{fig:2d-mag-hard-corr-xy}).

\begin{figure}[H]
	\centering
		\includegraphics[width=0.8\linewidth]{rys/ScanBot-10-2d-calibration-theta-sigma-2-added-hard-offset-reset-data-so-soft-iron-values-are-proper.PNG}
	\caption{Korekcja hard iron}
	\label{fig:2d-mag-hard-corr-xy}
\end{figure}


Teraz można przystąpić do procedury kompensacji zniekształceń \emph{soft iron}. Idealnie, po dokonaniu korekcji w dwóch wymiarach, punkty na wykresie można by umieścić na okręgu - tak jednak nie jest. Uzyskany kształt bardziej przypomina elipsę, i ta właściwość zostanie wykorzystana. Proces przebiega w kilku krokach:

\begin{enumerate}
    \item Dla każdego z punktów liczony jest promień $r$ (wektor od $(0,0)$ do danego punktu)
    \item Wyznaczana jest maksymalna wartość promienia $rmin$ i $rmax$. Te wartości odpowiadają kolejno półosi małej i półosi wielkiej elipsy.
    \item Dla punktu odpowiadającego promieniowi $rmax$ Obliczany jest kąt $\theta$ za pomocą dwuargumentowej funkcji $arctan2(y,x)$, gdzie $x$ i $y$ są współrzędnymi punktu. Kąt ten odpowiada kątowi obrotu elipsy.
    \item Tworzona jest macierz $R$. Posłuży do obrotu elipsy.
    $R = \begin{bmatrix}
            cos\theta & sin\theta\\
            -sin\theta & cos\theta
        \end{bmatrix}$
    \item Obliczany jest parametr $\sigma$. Posłuży do ściskania elipsy. $\sigma = \frac{rmin}{rmax}$
    \item Za pomocą macierzy $R$ elipsa jest obracana o kąt -$\theta$ w celu zrównania jej wielkiej półosi z osią OX układu współrzędnych. Współrzędne każdego z punktów kolejno są osobno przekształcane za pomocą mnożenia macierzy:
    $   
        \mathit{v}_{2}
        =
        \begin{bmatrix}
            cos\theta & sin\theta\\
            -sin\theta & cos\theta
        \end{bmatrix}
        \times
        \begin{bmatrix}
            \mathit{v}_{1x}\\
            \mathit{v}_{1y}
        \end{bmatrix}
    $
    gdzie $\mathit{v}_{2}$ to wynikowy wektor zawierający współrzędne przekształconego punktu.
    \item Obróconą elipsę sprowadza się do postaci okręgu poprzez ściśnięcie jej w osi X. Robi się to przemnażając współrzędną x każdego z punktów osobno przez wcześniej obliczony parametr: $\mathit{v}_{2x} = \mathit{v}_{1x} \times \sigma$
\end{enumerate}

Ostatecznie chmura punktów tworzy okrąg co przedstawiono na Rys. \ref{fig:2d-mag-soft-corr-applied}. Efektem zastosowania kalibracji jest równomierny odczyt azymutu wraz z obrotem platformy. Każdy zmierzony i przefitrowany punkt może być teraz przeliczony na kąt wektora, który wskazuje azymut - wystarczy skorzystać z dwuargumentowej funkcji $arctan2(x,y)$, gdzie $x$ i $y$ są współrzędnymi punktu.

\begin{figure}[ht]
	\centering
		\includegraphics[width=0.8\linewidth]{rys/ScanBot-11-2d-set-theta-then-sigma-and-done.PNG}
	\caption{Chmura punktów po dokonaniu korekcji obu typów zniekształceń}
	\label{fig:2d-mag-soft-corr-applied}
\end{figure}

Finalna wersja aplikacji korzysta z dwóch osi magnetometru. W celu kalibracji należy skorzystać z zakładki \emph{magnetometer calibration} głównego okna aplikacji sterującej. Rys. \ref{fig:main-app-mag-section-bottom} przedstawia najważniejsze elementy tej zakładki, wykorzystywane podczas półautomatycznego procesu kalibracji.

\begin{figure}[ht]
	\centering
		\includegraphics[width=1\linewidth]{rys/main-app-view-magnetom-bottom.png}
	\caption{Najważniejsze elementy sekcji kalibracji}
	\label{fig:main-app-mag-section-bottom}
\end{figure}

Aby dokonać kalibracji, robot powinien być połączony z aplikacją sterującą. Dalej procedura przebiega jak następuje:

\begin{enumerate}
    \item Należy przejśćdo zakładki \emph{magnetometer calibration}
    \item Jeżeli na wykresie znajdują się poprzednie pomiary, należy kliknąć przycisk \emph{CLEAR GRAPH DATA} aby je usunąć
    \item Należy wyzerować dane kalibracyjne znajdujące się w pamięci EEPROM robota. Służy do tego przycisk \emph{SEND CLEAR CALIBRATION DATA}
    \item Na tym etapie istnieją dwie możliwości przeprowadzenia pomiaru - za pomocą przycisku \emph{MEASURE [AUTO]} platforma samodzielnie, krokowo, wykona obrót wokół własnej osi i zbierze serię pomiarów z magnetometru; za pomocą przycisku \emph{MEASURE [MANUAL]} dokona jedynie pomiarów, w tym czasie należy robota obracać ręcznie. Postęp pomiarów wizualizowany jest na pasku postępu w prawym dolnym rogu. Na wykresie pojawią się zebrane pomiary. W sekcji oznaczonej \emph{HARD IRON CORRECTION} pojawiać się będą uzyskane dane dotyczące korekcji pierwszego z omawianych wcześniej zaburzeń - wartości minimalne i maksymalne dla każdej z osi oraz wartość przesunięcia (\emph{OFFSET}). W sekcji \emph{SOFT IRON CORRECTION} pojawiać się będą cyklicznie przeliczane wartości dotyczące korekcji zaburzeń \emph{soft iron} - m.in. parametry $\theta$ $\sigma$ i wartości macierzy $R$.
    \item Za pomocą przycisku \emph{CORRECT HARD IRON OFFSET} należy dokonać korekcji zaburzeń typu \emph{hard iron} na podstawie obliczonych wartości przesunięcia. Efekt natychmiastowo ukaże się na wykresie.
    \item Za pomocą przycisku \emph{CORRECT SOFT IRON OFFSET} należy dokonać korekcji zaburzeń typu \emph{soft iron}. Dane również brane są z obliczonych podczas procedury pomiaru. W tym momencie punkty na wykresie powinny być ułożone w okrąg. Jeżeli jest inaczej, oznacza to że w otoczeniu występują zaburzenia pola magnetycznego. Wtedy należy przemieścić robota w inne miejsce i powtórzyć wymienione czynności od nowa.
    \item Na koniec, aby wysłać dane kalibracyjne do robota, należy kliknąć przycisk \emph{SEND CALCULATED CALIBRATION DATA}. 
\end{enumerate}

Po dokonaniu procedury kalibracji pomiary dużo lepiej oddają rzeczywisty kierunek i zwrot platformy, jednak nawet podczas postoju platformy zwracana wartość azymutu nie pozostaje stała. Aby zniwelować to zjawisko konieczne jest zastosowanie filtru.

\subsubsection{Filtr Kalmana}
W układach sensorycznych robotów powszechnie wykorzytywany jest filtr Kalmana\cite{Kedzierski2016}. W tej sekcji wyjaśniona będzie zastosowana w projekcie, uproszczona implementacja takiego filtru.

\begin{lstlisting}[basicstyle=\footnotesize\ttfamily,language=python,caption=Implementacja filtru Kalmana w języku Python,label=lst:kalman]
def get_azimuth_kalman(self):
    q = 0.1
    estimate_err = 3
    measure_err = 3
    last_est = int(self.send("GET_AZIMUTH#"))

    for _ in range(20):
        measurement = int(self.send("GET_AZIMUTH#"))
        kalman_gain = estimate_err/(estimate_err + measure_err)
        current_est = last_est + kalman_gain*(measurement - last_est)
        estimate_err = (1 - kalman_gain)*estimate_err + abs(last_est-current_est)*q
        last_est = current_est
        ... (pominięto)
        
    current_est = int(current_est)
    self.azimuth = current_est
    self.publish_ros_odometry()
    return self.azimuth
\end{lstlisting}

Ze względu na ograniczone zasoby mikrokontrolera, filtracja odbywa się po stronie aplikacji sterującej. Pomiar dokonywany jest podczas gdy platforma stoi nieruchomo, stąd wiadomo że estymowana wartość jest stała. Na początku potrzebne będzie kilka odgórnie ustalonych parametrów. Wartości \emph{estimate\_err} i \emph{measure\_err} odpowiadają kolejno wariancji estymatora i wariancji mierzonej wartości (normalnie wyznaczone na podstawie dokładności pomiaru przyrządu), jednak nie muszą one być ustalone na podstawie rzeczywistych parametrów urządzenia. Ich wartości będą wpływać na działanie filtru - to czy dane będą bardziej "wygładzane", czy bardziej zależne od ostatniego pomiaru. W tym wypadku zostały wyznaczone eksperymentalnie. Parametr \emph{q} nie był konieczny, służy on regulacji zwiększenia wariancji estymaty w przypadku gdy estymowana wartość jest zmienna w czasie. Został on również dobrany eksperymentalnie. Chociaż ma mały wpływ na pomiary statyczne - jest zaimplementowany na potrzeby przyszłego rozwoju projektu.

Kolejną rzeczą która jest potrzebna jest aktualna wartość ostatniej estymacji \emph{last\_estimate}. Przy odpowiednio długim pomiarze tę wartość można ustawić na dowolną, jednak aby wartość estymatora szybciej zbiegała do estymowanej wstępnie zostaje ustawiona na zmierzoną wartość.

Mając wszystkie parametry można przystąpić do uruchomienia filtru. Przebiega on w 20 iteracjach. W każdej z iteracji zachodzą dwie fazy - predykcji i korekcji.
\\

Faza predykcji polega na wyznaczeniu wartości oczekiwanej apriori $\hat{x}(t+1)^-$ i odchylenia standardowego apriori $\sigma^2(t+1)^-$ dla czasu $t+1$ na podstawie analogicznych wartości aposteriori dla czasu $t$, tj.  $\hat{x}(t)^+$ i $\sigma^2(t)^+$. W tym wypadku z założenia wartość mierzona jest stała, dlatego przyjmuje się że:
\begin{center}
    $\hat{x}(t+1)^- = \hat{x}(t)^+$ \\
    $\sigma^2(t+1)^- = \sigma^2(t)^+$.    
\end{center}
Kod programu korzysta ze zmiennych \emph{est\_error} i \emph{current\_est}. Wartości apriori i aposteriori nie muszą być trzymane jednocześnie w pamięci - wystarczy nadpisywać stare zmienne. Tutaj oznaczałoby to sformułowanie konstrukcji \emph{est\_error}=\emph{est\_error} i \emph{current\_est}=\emph{current\_est} co nie jest potrzebne i dlatego zostało pominięte.
\\

Faza korekcji odbywa się w pętli. Najpierw dokonywany jest pomiar z sensora (\emph{measurement}). Następnie obliczane jest wzmocnienie Kalmana (\emph{kalman\_gain}), czyli parametr wpływający na to jakie znaczenie nowy pomiar ma dla wartości estymatora oraz jego wariancji. Później, z jego wykorzystaniem obliczana jest nowa wartość estymatora (\emph{current\_est}). W kolejnej linijce aktualizowana jest wartość jego wariancji (\emph{estimate\_err}). Aktualnie wraz z każdym nowym przejściem pętli zbiega ona do niezerowej wartości, zależnej od parametru q. Gdyby ten parametr nie zaistniał, z każdą iteracją jej wartość zbiegałaby do zera. Na koniec wartość (\emph{current\_est}) kopiowana jest do (\emph{last\_est}). Dzieje się to, ponieważ aktualizacja (\emph{estimate\_err}) potrzebuje przy obliczeniach wartości aktualnej oraz poprzedniej estymatora.

Po dwudziestokrotnym przejściu, wartość oczekiwana estymatora konwertowana jest na liczbę całkowitą, i zapisywanam, dane o azymucie są publikowane na odpowiednim temacie i wartość jest zwracana przez funkcję.


\subsection{Skan otoczenia i budowa mapy}
\label{sec:scan}
%>TODO powiedz o tym jak sie skaszaniło przy przewodzie pod podłogą<<<<<<<<<<<<<< koniecznie tutaj
%>TODO tu napisac o bazowym mapowaniu pkt na plaszczyzne
%o swojej implementacji ze scorem
%nastepnie o tym co moznaby dalej (translacja korekta)
%ale nie zostalo zrobione bo bylo slabe i to wymaga filtrow czasteczek
%i ze jest cos takiego jak cartographer od googla
%ale zdecydowano ze gmapper jest spoko i czemu nie gmapper

Samo zebranie pomiarów i przedstawienie ich w formie czytelnej dla człowieka nie jest trudne w realizacji. Pozyskane dane można przedstawić na wykresie, punkty połączyć prostymi liniami. Problem zaczyna się z agregacją wielu pomiarów, i na tym będzie koncentrował się ten rozdział.

Mając do dyspozycji dane z sensora \emph{LIDAR} oraz enkoderów i magnetometru, można rozpocząć proces skanowania. Pierwsze podejście oparte było o skan manualny, bez zliczania ścieżki - jedynie obrót był uwzględniony a dane ze skanera obracane o zczytany z sensora azymut. Robot został ustawiony w początkowej pozycji, uruchomiono procedurę skanowania za pomocą funkcji \emph{SCAN}, następnie z racji że skan zachodzi w zakresie kątów $\langle0^{\circ},180^{\circ}\rangle$, obrócony o $180^{\circ}$ (funkcja \emph{ROTATE}) po czym ponownie wykonano skanowanie. Efekt przedstawia Rys.\ref{fig:first-scan}. Jak widać, ściany pokoju (górna i dolna część rysunku) nie są równoległe. W tym momencie okazało się, że konieczna będzie kalibracja magnetometru, omówiona w rozdziale \ref{sec:odometry}. Po dokonaniu procedury kalibracji czynności powtórzono, rezultat ukazano na Rys. \ref{fig:first-magnetom-calibrated-scan}. Na dalej przedstawianych rysunkach skany mogą wyglądać nieco inaczej, głównie przez zamknięte lub otwarte drzwi pokoju - widoczne jest to na ostatnich dwóch wspomnianych rysunkach jako najdłuższe z promieni - w tym wypadku drzwi były otwarte a pomiar uwzględniał fragment przedpokoju autora. Skany przedstawiane są w zakładce \emph{map} okna głównego aplikacji sterującej.

\begin{figure}[ht]
	\centering
		\includegraphics[width=0.5\linewidth]{rys/ScanBot-01-room-map-nocalibration.png}
	\caption{Pierwszy skan pokoju}
	\label{fig:first-scan}
\end{figure}

\begin{figure}[ht]
	\centering
		\includegraphics[width=0.5\linewidth]{rys/ScanBot-02-room-map-calibrated.png}
	\caption{Skan pokoju po zastosowaniu kalibracji magnetometru}
	\label{fig:first-magnetom-calibrated-scan}
\end{figure}

%>TODO pisz o dopasowaniu



\begin{figure}[ht]
	\centering
		\includegraphics[width=0.5\linewidth]{rys/ScanBot-12-calibrated-room-map1.PNG}
	\caption{>TODO}
	\label{fig:}
\end{figure}

\begin{figure}[ht]
	\centering
		\includegraphics[width=0.5\linewidth]{rys/ScanBot-12-calibrated-room-map2.PNG}
	\caption{>TODO}
	\label{fig:xxx}
\end{figure}

\begin{figure}[ht]
	\centering
		\includegraphics[width=0.5\linewidth]{rys/ScanBot-12-calibrated-room-map3.PNG}
	\caption{>TODO}
	\label{fig:xxx}
\end{figure}

\begin{figure}[ht]
	\centering
		\includegraphics[width=0.5\linewidth]{rys/ScanBot-12-calibrated-room-map4.PNG}
	\caption{>TODO}
	\label{fig:xxx}
\end{figure}


%>TODO teraz o pierwotnym dopasowaniu ze scorem
%>TODO zdjecia do tg


%>TODO teraz zmiana na RViz

%>TODO napisz o interferencji

%%przewod - interferencja
\begin{figure}[ht]
	\centering
		\includegraphics[width=0.8\linewidth]{rys/calibrated-mag-high-interference-broken-rotation.PNG}
	\caption{>TODO}
	\label{fig:xxx}
\end{figure}

%% znowu przewod
\begin{figure}[ht]
	\centering
		\includegraphics[width=0.8\linewidth]{rys/2020-11-04-170347_1920x1080_scrot.PNG}
	\caption{>TODO}
	\label{fig:xxx}
\end{figure}


%>TODO odometria jest słaba, nie pozwala na skan całego domu. Na ten moment obszar ograniczony został do pojedynczego pokoju.

%TODO skany z wykorzystaniem algorytmu jazdy autonomicznej (mozna ew. dorobić ścieżkę)





\subsubsection{UMBenchmark\cite{Borenstein1995}}
%>TODO tu dac fotki koniecznie
%jakies rysunki nabazgrane z kwadratami
% i powiedziec ze sie nie pokrywalo najlepiej
% dac tabelke z pozycjami
% pokazac obliczenia center of gravity itd
% wyszlo ~1proc czyli to bez sensu
% wiec robot juz jezdzi wystarczajaco dobrze

% >TODO nawiaz do wczesniej wspomnianych rzeczy
% powiedz co po kolei jak sie zmieniało
% z czym były problemy
% o tutaj mala podpowiedz:
% sensor dzwieku jest do kitu za duza fluktuacja pluz ogromny kąt plus za duzy blad pomiaru plus za dlugi pomiar
% sharp mniejszy kąt mniejsza fluktuacja krotki pomiar ale podatnosc na swiatlo
% lidar super miodzio
% serwo kiepskie i wibruje/zacina się co powoduje błędy na krawędziach
% magnetometr-biblioteka koryguje jedynie hard iron i jest do kitu
% ze wzgledu na moc obliczeniowa najlepiej bylo samemu zaimplementowac soft i hard iron
% zeby bylo prosciej tylko 2 wymiary, niestety nie kompensujemy tiltu. ale i tak zalozenie jest ze jezdzimy po plaskim

% używanie arcsin zamiast atan2 z ifami podowowało błąd że correction czasem obracał elipsę nie w tę stronę
\chapter{Podsumowanie}
\section{Ocena własna projektu}
>TODO uzupełnić, wyrazić zadowolenie z wyników przeprowadzonych doświadczeń, szczególnie UMBmark który wykazał że projekt podwozia jest dobry

\section{Wnioski}
>TODO uzupełnić

%\bibliographystyle{plalpha}
\bibliographystyle{plabbrv}

\setlength{\bibitemsep}{2pt}
\bibliography{my-bibliography.bib}
\appendix
\chapter{Opis załączonej płyty CD/DVD}
%>TODO opisać co znajduje się na płycie CD



\chapterstyle{noNumbered}
\phantomsection
\addcontentsline{toc}{chapter}{Indeks rzeczowy}
\printindex

\end{document}


% 1. Preferowany rozmiar czcionki to 11 lub 12 pkt (mniejsza czcionka może być zastosowana do podpisów rysunków/tabel/równań), interlinia 1.0 lub 1.5.

% 2. Każdy zamieszczony element typu rysunek/tabela/równanie musi mieć swój numer. Rysunki i tabele muszą mieć ten opis.

% 3. Do elementów typu rysunek/tabela/równanie odwołujemy się za pomocą jego numeru, czyli np. "Rysunek 1 przedstawia...". Nie stosujemy sformułowań typu "na rysunku poniże/powyżej".

% 4. Podpis/opis tabeli zamieszczamy nad tabelą, podpis/opis rysunku - pod rysunkiem. Numery równań podajemy w tej samej linii co równanie, po prawej stronie.

% 5. W pracy musi znajdować się bibliografia, a każda pozycja zawarta w bibliografii musi być przynajmniej raz zacytowana. Do cytowania używamy nawiasów klamrowych, w których znajduje się klucz (najczęściej numer) określonej pozycji bibliograficznej. Np. [1] czy też [10].

% 6. Pisząc pracę pamiętamy o ciąg przyczynowo-skutkowym oraz tym, że czytelnik na etapie i-tego rodziału nie wie co przedstawione jest w rozdziale (i+1).

% 7. Każdy akronim użyty w pracy musi zostać zdefiniowany w miejscu jego pierwszego wystąpienia w tekście. Jeśli używanych akronimów jest wiele warto za początku pracy zamieścić ich spis.

% 8. Rysunki zamieszczone w pracy powinny być Państwa autorstwa. Przy każdym rysunku wzięty z literatury należy zamieścić referencję do pozycji bibliograficznej, z której ten rysunek pochodzi.
\chapter*{Słownik skrótów i pojęć}\mbox{}\pdfbookmark[0]{Słownik skrótów i pojęć}{skroty.1}
\label{sec:skroty}
\noindent
\begin{description}[labelwidth=*]
  \item  [IoT] (ang. \emph{Internet of Things}) \emph{Internet Rzeczy} - koncepcja przedmiotów codziennego użytku samodzielnie komunikujących się za pośrednictwem sieci Internet bądź innego środka w celu gromadzenia i przetwarzania danych
  
  \item [ToF] (ang. \emph{Time of Flight}) \emph{Czas lotu} - tutaj w odniesieniu do rodzaju sensora laserowego mierzącego dystans za pomocą pomiaru czasu jaki upływa od wysłania impulsu świetlnego do odebrania jego odbicia od przeszkody
  
  \item [SLAM] (ang. \emph{Simultaneous Localization And Mapping}) \emph{Symultaniczna lokalizacja i mapowanie} - problem obliczeniowy dotyczący jednoczesnego tworzenia obrysu obiektów w otoczeniu na podstawie pomiarów i znanej lokalizacji jednostki mierzącej w przestrzeni oraz określenia położenia jednostki mierzącej w otoczeniu na podstawie dokonywanych przez nią pomiarów
  
  \item [VSLAM] >TODO uzupełnić
  
  \item [LIDAR] >TODO uzupełnić
  
  \item [USB-UART] >TODO uzupełnić
  
  \item [PC] (ang. \emph{Personal Computer}) \emph{Komputer Osobisty}
  
  \item [smart] (z ang. \emph{sprytny})  - w kontekście tego dokumentu jest to jedno z haseł często występujących w nazwach i reklamach produktów wyposażonych w dowolnego rodzaju mikrokomputer z oprogramowaniem pozwalającym na komunikację sprzętu z aplikacją na smartfona bądź interfejsem przeglądarkowym, rzekomo ułatwiającym korzystanie z niego. Ubarwia ono obraz urządzenia w oczach konsumenta mając przekonać go do jego innowacyjności. 
  
  \item [intelligent] (z ang. \emph{inteligentny})  - podobnie jak ``smart`` w kontekście niniejszej pracy jest często wykorzystywanym określeniem nowoczesnych produktów
  
  \item [Firmware] - oprogramowanie wbudowane działające na mikrokontrolerze
  
  \item [Arduino] - rodzaj płytek drukowanych z wbudowanymi mikrokontrolerami o publicznie dostępnych schematach elektronicznych, również otwartoźródłowe środowisko służące do pisania, kompilacji i wgrywania programów na płytki
  
  \item [STM32duino] - nieoficjalna implementacja środowiska kompilacji (bibliotek) Arduino dla płytek z mikrokontrolerami STM32
  
\end{description}

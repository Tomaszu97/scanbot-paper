\chapter{Projekt}
\section{Oprogramowanie PC}
\subsection{Struktura oprogramowania}
>TODO dac jakies UMLe itd

\subsection{Autonomiczna jazda}
>TODO tu napisać o swoim algorytmie, dac rysunek stref

Algorytm autonomicznej jazdy został napisany po stronie aplikacji sterującej na komputerze PC. Kieruje się on bardzo prostymi regułami i pozwala na samodzielne poruszanie się platformy w warunkach domowych, omijania przeszkód i zgrubnego podążania za ścianami bądź też w pobliżu innych przeszkód.

Zasada działania algorytmu jest bardzo prosta i opiera się na wykrywaniu przeszkód w wyznaczonych strefach i decyzji o podjętych ruchach.

\subsection{Skan otoczenia}
>TODO tu napisac o bazowym mapowaniu pkt na plaszczyzne
o swojej implementacji ze scorem
nastepnie o tym co moznaby dalej (translacja korekta)
ale nie zostalo zrobione bo bylo slabe i to wymaga filtrow czasteczek
i ze jest cos takiego jak cartographer od googla
ale zdecydowano ze gmapper jest spoko i czemu nie gmapper

\subsection{Odometria}
\subsubsection{Kalibracja magnetometru}
>TODO koniecznie daj screenshoty
opisz hard i soft iron offset
powiedz o filtrze kalmana

\subsubsection{UMBenchmark\cite{Borenstein1995}}
>TODO tu dac fotki koniecznie
jakies rysunki nabazgrane z kwadratami
i powiedziec ze sie nie pokrywalo najlepiej
dac tabelke z pozycjami
pokazac obliczenia center of gravity itd
ej wyszlo ~1% czyli to bez sensu
robot juz jezdzi dobrze
to zasługa wielkiego myśliciela autora tego tekstu
oraz komitetu centralnego
a jakże

\section{Firmware}
>TODO powiedziec o ograniczeniach wynikajacych ze stosowania
mikrokontrolera, o srodowisku stm z ktorego nie skorzystalem
o srodowisku stm32duino o tym ze stm32 jest szybsze i wgl
super bo to wszystko zre mniej pradu niz rasp i nie ma problemu kiedy zasilanie spada

\section{Schemat budowy mechanicznej}
>TODO powiedz co sie zmienialo, jaki jest problem z gasienicami
wyjasnij ze gasienice upraszczamy do 2 kolek

\section{Schemat elektroniczny}
>TODO wspomnij o sensorach ultradzw oraz sharp IR
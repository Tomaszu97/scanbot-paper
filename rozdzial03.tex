\chapter{Ewaluacja}

\subsection{Skan otoczenia}
%>TODO tu napisac o bazowym mapowaniu pkt na plaszczyzne
%o swojej implementacji ze scorem
%nastepnie o tym co moznaby dalej (translacja korekta)
%ale nie zostalo zrobione bo bylo slabe i to wymaga filtrow czasteczek
%i ze jest cos takiego jak cartographer od googla
%ale zdecydowano ze gmapper jest spoko i czemu nie gmapper

\subsection{Odometria}
\subsubsection{Kalibracja magnetometru}
\label{sec:mag_cal}
%>TODO koniecznie daj screenshoty
%opisz hard i soft iron offset
%powiedz o filtrze kalmana

\subsubsection{Filtr Kalmana}
\label{sec:kalman}

\subsubsection{UMBenchmark\cite{Borenstein1995}}
%>TODO tu dac fotki koniecznie
%jakies rysunki nabazgrane z kwadratami
% i powiedziec ze sie nie pokrywalo najlepiej
% dac tabelke z pozycjami
% pokazac obliczenia center of gravity itd
% wyszlo ~1proc czyli to bez sensu
% wiec robot juz jezdzi wystarczajaco dobrze

% >TODO nawiaz do wczesniej wspomnianych rzeczy
% powiedz co po kolei jak sie zmieniało
% z czym były problemy
% o tutaj mala podpowiedz:

% sensor dzwieku jest do kitu za duza fluktuacja pluz ogromny kąt plus za duzy blad pomiaru plus za dlugi pomiar
% sharp mniejszy kąt mniejsza fluktuacja krotki pomiar ale podatnosc na swiatlo
% lidar super miodzio
% serwo kiepskie i wibruje/zacina się co powoduje błędy na krawędziach
% magnetometr-biblioteka koryguje jedynie hard iron i jest do kitu
% ze wzgledu na moc obliczeniowa najlepiej bylo samemu zaimplementowac soft i hard iron
% zeby bylo prosciej tylko 2 wymiary, niestety nie kompensujemy tiltu. ale i tak zalozenie jest ze jezdzimy po plaskim

% używanie arcsin zamiast atan2 z ifami podowowało błąd że correction czasem obracał elipsę nie w tę stronę
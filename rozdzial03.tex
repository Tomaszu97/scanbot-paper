\chapter{Ewaluacja}
>TODO wymysl jakies podrozdziały tutaj

>TODO nawiaz do wczesniej wspomnianych rzeczy
powiedz co po kolei jak sie zmieniało
z czym były problemy
o tutaj mala podpowiedz:

sensor dzwieku jest do kitu za duza fluktuacja pluz ogromny kąt plus za duzy blad pomiaru plus za dlugi pomiar
sharp mniejszy kąt mniejsza fluktuacja krotki pomiar ale podatnosc na swiatlo
lidar super miodzio
serwo kiepskie i wibruje/zacina się co powoduje błędy na krawędziach
magnetometr-biblioteka koryguje jedynie hard iron i jest do kitu
ze wzgledu na moc obliczeniowa najlepiej bylo samemu zaimplementowac soft i hard iron
zeby bylo prosciej tylko 2 wymiary, niestety nie kompensujemy tiltu. ale i tak zalozenie jest ze jezdzimy po plaskim

używanie arcsin zamiast atan2 z ifami podowowało błąd że correction czasem obracał elipsę nie w tę stronę
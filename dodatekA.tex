\chapter{Opis załączonej płyty CD/DVD}
%TODO nagraj płytę i zmień nagłówek na CD albo DVD w zależności od wyboru
\label{sec:disc-addon}
Na załączonej płycie znajdują się 3 foldery:
\begin{enumerate}
    \item 3d-model - w tym folderze znajduje się trójwymiarowy projekt robota wykonany w programie Autodesk Inventor.
    \item src-firmware - w tym folderze znajduje się kod źródłowy oprogramowania platformy napisany w środowisku \emph{PlatformIO} \cite{platformio}.
    \item src-pc - folder zawierający moduł \emph{scanbot\_communicator} w którym znajduje się aplikacja sterująca.
\end{enumerate}

\noindent \textbf{Ad.1} \\  

Plik \emph{Assembly1.ipm} zawiera projekt podwozia wraz z górną pokrywą i kołami. \\ 

Plik \emph{arm.ipt} to projekt wieżyczki na której zamieszczony jest sensor laserowy. \\ 

Plik \emph{tracks\_v2.ipt} zawiera projekt gąsienicy. \\[10pt]

\noindent \textbf{Ad.2} \\ 

Cały folder należy otworzyć w środowisku \emph{PlatformIO}. \\ 
Główny plik z programem (\emph{main.cpp}) znajduje się w podfolderze \emph{src}. \\[10pt]

\noindent \textbf{Ad.2} \\ 

Właściwy moduł \emph{ROS} znajduje się w podfolderze \emph{scanbot\_communicator}. Instrukcje dotyczące korzystania, w języku angielskim, znajdują się w pliku \emph{README.md}. Dodatkowo plik \emph{bag1.bag} zawiera nagranie komunikacji zachodzącej za pośrednictwem tematów \emph{ROS} podczas przejazdu robota, wykorzystywane przy budowaniu ostatniej z przedstawionych w niniejszym dokumencie map. Instrukcja w jaki sposób odtworzyć to nagranie znajduje się we wspomnianym pliku \emph{README.md}.